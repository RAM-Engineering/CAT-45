\documentclass[12pt]{article}
\usepackage{graphicx}
\usepackage{amsmath}
\setlength\parindent{0pt}

\begin{document}

\title{Conical Axisymmetric Thruster 45}
\author{Arsh Noman\\\ \\\ Carolina Skylab}
\date{December 2025}



\maketitle
\thispagestyle{empty}
\newpage

\thispagestyle{empty}
\tableofcontents
\thispagestyle{empty}
\newpage


\section{Introduction}

This document presents the conceptual development and preliminary engineering design of Carolina Skylab’s first internally developed rocket engine, designated the \texttt{Conical Axisymmetric Thruster 45 (CAT-45)}. The CAT-45 is a small solid propellant rocket motor incorporating an axisymmetric conical nozzle fabricated from graphite, developed as an educational and experimental propulsion system within a university aerospace engineering club under academic and industry supervision.

The objective of this work is to demonstrate a structured and defensible design methodology grounded in classical rocket propulsion theory rather than empirical trial and error. The CAT-45 is designed to deliver a total impulse of 320~Ns over a minimum burn duration of five seconds, while maintaining realistic chamber pressure, stable combustion behavior, and structurally acceptable operating margins. Emphasis is placed on analytical consistency between thrust requirements, nozzle geometry, chamber pressure selection, and solid propellant burning characteristics, in accordance with accepted professional practice.

The analytical framework and design logic used throughout this document are based primarily on the work of George P. Sutton and Oscar Biblarz, as presented in \texttt{Rocket Propulsion Elements, Seventh Edition}. This text is treated as the authoritative reference for thrust relations, nozzle flow theory, solid rocket motor fundamentals, and systematic motor design methodology. All governing equations, assumptions, and design decisions are derived from or justified using this reference to ensure academic rigor and technical credibility.

The CAT-45 motor is conceptually based on a potassium nitrate and sorbitol solid propellant formulation, selected for its well documented combustion behavior, moderate flame temperature, and suitability for instructional analysis. This propellant system enables clear analytical coupling between chamber pressure, burning rate, and mass flow while remaining compatible with graphite nozzle materials under conservative operating assumptions. Fabrication of the propulsion hardware relies on conventional university laboratory equipment, including TIG welding for the motor casing and a manual lathe with a taper attachment for machining the axisymmetric conical nozzle. The project follows a four month development timeline, progressing from analytical sizing and design validation to fabrication, assembly, inspection, and documentation, with deliberate allowance for review and iterative refinement.

\newpage

\section{Design Objectives and Requirements}

The design of the Conical Axisymmetric Thruster 45 (CAT-45) is governed by a set of performance driven and practical objectives intended to reflect realistic solid rocket motor design practice within an academic engineering environment. The primary objective is to develop a self consistent preliminary design for a small solid propellant rocket motor that satisfies the H-class of specified impulse while remaining analytically traceable, structurally credible, and suitable for fabrication using conventional university laboratory equipment. The design emphasizes theoretical closure between propulsion performance, nozzle geometry, chamber pressure selection, and propellant burning behavior, rather than optimization for maximum performance.

A secondary objective is to ensure that all major design decisions can be justified using established rocket propulsion theory as presented in standard references, with particular emphasis on solid rocket motor fundamentals and nozzle flow theory. The resulting design is intended to serve as both a learning platform and a foundation for future propulsion development efforts within Carolina Skylab, rather than as a flight qualified or production ready motor.

\subsection{Total Impulse and Burn Time Requirements}

The CAT-45 motor is designed to deliver a total impulse of 320~Ns, corresponding to the highest boundary of an H class solid rocket motor. Total impulse is defined as the time integral of thrust over the burn duration and represents the primary performance metric for classification and sizing of solid rocket motors. For preliminary design purposes, thrust is assumed to be approximately constant over the burn interval, allowing the total impulse requirement to be expressed as the product of average thrust and burn time.

A minimum burn duration of five seconds is imposed as a design requirement in order to limit peak thrust levels, reduce structural loading, and promote stable combustion behavior. This constraint directly influences the selection of chamber pressure, throat area, and grain geometry. By specifying both total impulse and minimum burn time, the average thrust level is bounded, providing a clear starting point for analytical sizing of the nozzle and combustion chamber. Hence, the required average thrust is
\[F_{\text{avg}} = \frac{320~\text{N}\cdot\text{s}}{5~\text{s}} =64~\text{N}\]

\subsection{Performance Constraints and Design Assumptions}

Several performance constraints and simplifying assumptions are adopted to enable a tractable and defensible preliminary design. Chamber pressure is treated as a design variable selected to balance achievable thrust, realistic propellant burning behavior, and structural limitations of the motor casing. Pressure transients associated with ignition are acknowledged but are not explicitly modeled, and operating pressure is assumed to remain near a steady nominal value during the majority of the burn.

The nozzle is assumed to operate under quasi steady, one dimensional, isentropic flow conditions with appropriate correction for real nozzle performance losses. The thrust coefficient and characteristic velocity are treated as representative values consistent with the selected propellant and nozzle geometry. Heat transfer, nozzle erosion, and long term material degradation effects are not explicitly modeled and are instead addressed through conservative operating assumptions and material selection.

The propellant is assumed to burn in a stable and predictable manner according to a pressure dependent burning rate law, with uniform regression over the exposed burning surfaces. Grain geometry is selected to provide an approximately neutral thrust profile over the burn duration. Structural analysis of the motor casing is based on thin wall pressure vessel theory with appropriate safety margin applied to account for material variability and manufacturing tolerances.

These constraints and assumptions define the scope of the present design and are consistent with the level of fidelity expected in a preliminary engineering analysis conducted within an academic setting.

\newpage

\section{Fundamental Rocket Performance Relations}

The preliminary design of the CAT-45 motor is founded on the fundamental performance relations governing chemical rocket propulsion. These relations establish the quantitative connection between thrust, propellant mass flow, chamber pressure, and nozzle geometry. In accordance with classical rocket theory, the motor is treated as a steady flow device during the majority of its operating interval, allowing time averaged performance parameters to be used for initial sizing and analysis.

\subsection{Thrust, Mass Flow, and Specific Impulse}

Thrust is generated by the acceleration of combustion products through the nozzle and is defined as the rate of change of momentum of the exhaust flow together with the pressure forces acting on the nozzle exit plane. For a chemical rocket operating at steady conditions, thrust may be expressed as
\[
F = \dot{m} v_e + (p_e - p_a) A_e
\]
where $\dot{m}$ is the propellant mass flow rate, $v_e$ is the effective exhaust velocity, $p_e$ is the nozzle exit pressure, $p_a$ is the ambient pressure, and $A_e$ is the nozzle exit area. A central performance metric for rocket engines is the specific impulse, defined as the thrust produced per unit weight flow rate of propellant,
\[
I_{sp} = \frac{F}{\dot{m} g_0}
\]
where $g_0$ is the standard gravitational acceleration. Specific impulse provides a measure of propulsion efficiency independent of engine size and is used to compare the performance of different propulsion systems. For solid rocket motors, the specific impulse is primarily influenced by propellant thermochemical properties and nozzle expansion characteristics. For preliminary design, the thrust of the CAT-45 motor is assumed to be approximately constant over the burn duration. Under this assumption, the total impulse requirement may be related directly to the average thrust and burn time, providing a clear link between performance objectives and mass flow rate.

These tools now make it possible to calculate the mass flow rate $\dot{m}$:

For a KN/Sorbitol propellant with a specific impulse of approximately$^{2}$
\[
I_{sp} \approx 151.5~\text{s}
\]
the total propellant mass is given by
\[
m_{\text{prop}} = \frac{I_{\text{total}}}{I_{sp}\, g_0}
\]
Substituting numerical values,
\[
m_{\text{prop}} = \frac{320}{151.5 \times 9.81} = 0.215~\text{kg} \approx 220~\text{g}.
\]

Hence, we can now see that CAT-45, in order to achieve the thrust metrics set by the project objectives, would require at least 215 grams of KN/Sorbitol propellant. We will design for a propellant mass of 220 grams, with the 5 grams as buffer mass. Hence, the corresponding mass flow rate is
\[\dot{m} = \frac{220~\text{g}}{5~\text{s}} = 44~\text{g/s} = 0.044~\text{kg/s}.\]


\subsection{Characteristic Velocity and Thrust Coefficient}

To decouple combustion performance from nozzle performance, classical rocket
theory introduces the characteristic velocity c*, defined as:
\[c^* = \frac{p_c A_t}{\dot{m}}\]
where $p_c$ is the chamber pressure and $A_t$ is the nozzle throat area. The characteristic velocity depends primarily on propellant composition and combustion efficiency and is largely independent of nozzle geometry. Hence, it serves as a reliable parameter for relating chamber conditions to mass flow rate in solid rocket motor analysis. The thrust coefficient $C_F$ is used to relate chamber pressure and throat area directly to thrust and is defined as:
\[C_F = \frac{F}{p_c A_t}\]

The thrust coefficient accounts for the conversion of thermal energy into directed exhaust momentum through the nozzle and incorporates the effects of gas expansion and ambient pressure. For a given propellant and operating environment, $C_F$ is determined primarily by nozzle geometry and expansion ratio. 

Together, the characteristic velocity and thrust coefficient allow thrust to be expressed in the compact form:
\[F = C_F p_c A_t\]

This relation forms the basis for preliminary nozzle throat sizing and chamber pressure selection in the CAT-45 design. From the definition of characteristic velocity, we can establish a fundamental
relationship that must be satisfied:
\[p_c A_t = \dot{m} c^* \]

For KN/Sorbitol propellant, the characteristic velocity from experimental data
and thermochemical calculations is approximately:${^2}$
\[c^* = 908~\text{m/s}\]

Substituting the required mass flow rate from Section 3.1:
\[p_c A_t = (0.044~\text{kg/s})(908~\text{m/s}) = 39.952~\text{N}\]

This establishes a fundamental constraint, the product of chamber pressure  and throat area must equal 39.952N to achieve the required mass flow rate of 44 g/s. The relationship $p_c A_t =39.952$ N represents one equation with two unknowns. To proceed with the design, we must select either chamber pressure or throat area based on additional constraints:\\\

1. \textbf{Structural constraints}: limit maximum chamber pressure based on motor casing material strength and wall thickness.\\\
2. \textbf{Manufacturing constraints}: limit minimum throat diameter based on machining capabilities.\\\
3. \textbf{Propellant burning behavior}: requires minimum chamber pressure for stable combustion.\\\

These constraints will be examined systematically in subsequent sections to
arrive at a consistent design solution. The thrust coefficient $C_F$ will be
determined once the nozzle geometry is fully specified in Section 4.

\newpage

\section{Nozzle Flow Theory and Geometry}

The nozzle is the primary component responsible for converting the thermal energy of the combustion gases into directed kinetic energy and thrust. In the CAT-45 design, the nozzle is treated as an axisymmetric, fixed geometry, conical nozzle operating under steady flow conditions. The analysis presented in this section follows classical one dimensional compressible flow theory and provides the basis for throat sizing, expansion ratio selection, and performance estimation.

\subsection{Isentropic Flow Relations}

For preliminary nozzle analysis, the flow through the nozzle is assumed to be steady, adiabatic, and one dimensional, with negligible body forces. Under these assumptions, the expansion of the combustion gases may be treated as isentropic except for localized losses. The governing isentropic relations relate pressure, temperature, density, and Mach number throughout the nozzle.

At the nozzle throat, the flow is choked and the Mach number is equal to unity. The mass flow rate through the nozzle is therefore fixed by the chamber pressure, throat area, and propellant gas properties. The ratio of static pressure to chamber pressure at any point in the nozzle is given by
\[
\frac{p}{p_c} =
\left(1 + \frac{\gamma - 1}{2} M^2 \right)^{-\frac{\gamma}{\gamma - 1}}
\]
where $p$ is the local static pressure, $p_c$ is the chamber pressure, $\gamma$ is the ratio of specific heats, and $M$ is the local Mach number. These relations provide the foundation for determining exit conditions as a function of nozzle geometry and are used to estimate exit pressure, exit velocity, and thrust contribution from pressure forces.

\subsection{Conical Nozzle Configuration}

The CAT-45 employs a conical nozzle geometry due to its simplicity, manufacturability, and suitability for small solid rocket motors. A conical nozzle consists of a converging section leading to a minimum area throat, followed by a straight walled diverging section defined by a constant half angle. While bell shaped nozzles offer higher performance for large expansion ratios, conical nozzles remain widely used in small scale applications because of their reduced fabrication complexity and acceptable performance penalties. The divergence angle represents a tradeoff between nozzle length and divergence losses. Larger angles reduce nozzle length but increase losses due to non axial exhaust velocity components. For CAT-45's preliminary design, the conical nozzle is treated as axisymmetric and flow is assumed to expand uniformly along the nozzle centerline. This assumption is consistent with classical nozzle analysis and is appropriate for the required level of fidelity.

\subsection{Expansion Ratio and Ambient Effects}

The nozzle expansion ratio is defined as the ratio of exit area to throat area:
\[\varepsilon = \frac{A_e}{A_t}\]
It is a primary parameter governing nozzle performance. For a given chamber pressure and propellant, the expansion ratio determines the exit Mach number and exit pressure through the isentropic flow relations presented in Section 4.1. Ideally, the nozzle exit pressure should equal the ambient pressure to maximize
thrust. This condition, known as perfect expansion, occurs when:
\[p_e = p_a\]

If the exit pressure exceeds ambient pressure ($p_e > p_a$), the nozzle is underexpanded and additional thrust potential is unused. If the exit pressure is below ambient pressure ($p_e < p_a$), the nozzle is overexpanded, which can lead to flow separation and performance loss. For the CAT-45 design operating at sea level, $\varepsilon$ must be selected to balance performance, nozzle length, and manufacturing complexity. The following are the environmental assumptions we will be utilizing throughout the calculation process:

\begin{itemize}
\item Ambient pressure: $p_a = 100{,}068$ Pa (Morrisville's average pressure every May for the past three years)$^3$
\item Chamber pressure: $p_c = 3.447 \times 10^6$ Pa (500 psi, justified in Section 5)
\item Molecular weight$^2$: $M \approx 39.9$ g/mol
\item Ratio of specific heats$^2$: $\gamma = 1.137$
\item Combustion temperature$^2$: $T_c \approx 1327$ K
\end{itemize}

For perfect expansion at sea level, we require $p_e = p_a$. The pressure ratio across the nozzle is:
\[\frac{p_e}{p_c} = \frac{100{,}068}{3.447 \times 10^6} = 0.0290\]

From the isentropic relation presented in Section 4.1:
\[
\frac{p}{p_c} = \left[1 + \frac{\gamma - 1}{2}M^2\right]^{-\frac{\gamma}{\gamma-1}}.
\]

Solving for exit Mach number when $p_e/p_c = 0.0290$:
\[
0.0290 = \left[1 + \frac{0.137}{2}M_e^2\right]^{-\frac{1.137}{0.137}}\]
\[0.0290 = \left[1 + 0.0685 M_e^2\right]^{-8.30}\]

Taking the reciprocal:
\[34.48 = \left[1 + 0.0685 M_e^2\right]^{8.30}\]

Taking the 8.30th root:
\[1 + 0.0685 M_e^2 = 34.48^{1/8.30} = 1.532\]

Solving for Mach number:
\[0.0685 M_e^2 = 0.532\]
\[M_e^2 = 7.77\]
\[M_e = 2.79\]

The area-Mach number relation for isentropic flow is
\[\frac{A}{A_t} = \frac{1}{M}\left[\frac{2}{\gamma+1}\left(1 + \frac{\gamma-1}{2}M^2\right)\right]^{\frac{\gamma+1}{2(\gamma-1)}}\]

For $M_e = 2.79$ and $\gamma = 1.137$:
\[
\varepsilon_{\text{ideal}} = \frac{A_e}{A_t} = \frac{1}{2.79}\left[\frac{2}{2.137}\left(1 + 0.0685 \times 7.77\right)\right]^{\frac{2.137}{2 \times 0.137}}\]
\[\varepsilon_{\text{ideal}} = \frac{1}{2.79}\left[0.936 \times 1.532\right]^{7.80}\]
\[\varepsilon_{\text{ideal}} = \frac{1}{2.79}(1.434)^{7.80}\]
\[\varepsilon_{\text{ideal}} = \frac{1}{2.79} \times 16.63\]
\[\varepsilon_{\text{ideal}} = 5.96\]

Hence, it becomes reasonable to claim an expansion ratio of $\varepsilon = 6$ as a result of the the calculations above.

\subsection{Nozzle Performance Losses}

Real rocket nozzles deviate from ideal one dimensional isentropic behavior due to several loss mechanisms. The primary contributors include viscous boundary layer losses, divergence losses associated with non axial exhaust flow, and imperfect pressure expansion at the nozzle exit. These effects reduce the actual thrust below the ideal isentropic prediction. For short conical nozzles with moderate expansion ratios, divergence loss is a significant and well characterized geometric effect. Divergence loss arises because the exhaust flow exits the nozzle at a finite angle relative to the nozzle centerline, so only the axial component of momentum contributes to thrust. For a conical nozzle with divergence half angle $\alpha$, assuming uniform exit velocity magnitude, uniform angular distribution, negligible flow separation, and a thin boundary layer, the divergence loss correction factor is given by:

\[\lambda = \frac{1 + \cos \alpha}{2}\]

Under these assumptions, the actual thrust coefficient is related to the ideal thrust coefficient by:

\[C_F = \lambda C_{F\text{ideal}}\]

where $C_{F\text{ideal}}$ corresponds to isentropic, perfectly expanded flow with purely axial exhaust velocity. The CAT-45 nozzle is a conical nozzle with a divergence half angle of $\alpha = 15^\circ$. Substituting into the divergence correction factor:
\[\lambda = \frac{1 + \cos(15^\circ)}{2} = \frac{1 + 0.9659}{2} = 0.983\]

This indicates a thrust reduction of:

\[1 - \lambda = 0.017\]

or approximately $1.7\%$, relative to an ideal nozzle with purely axial exhaust flow. The ideal thrust coefficient for the CAT-45 design is obtained from isentropic nozzle relations assuming perfect expansion at sea level, negligible viscous losses, and no divergence loss. Using the results derived in Section 4.3, for the selected operating conditions:
\[
\varepsilon = 6, \quad \gamma = 1.137, \quad p_a = 100{,}068~\text{Pa},
\]
the ideal thrust coefficient is obtained from the standard isentropic thrust coefficient expression$^4$:
\[
C_{F,\text{ideal}}
=
\sqrt{
\frac{2\gamma^2}{\gamma-1}
\left(\frac{2}{\gamma+1}\right)^{\frac{\gamma+1}{\gamma-1}}
\left[
1 - \left(\frac{p_e}{p_c}\right)^{\frac{\gamma-1}{\gamma}}
\right]
}
+
\left(\frac{p_e - p_a}{p_c}\right)\varepsilon
\]

For ideal (perfectly expanded) operation at sea level, the exit pressure equals the ambient pressure:
\[
p_e = p_a
\]
so the pressure thrust term vanishes and the expression reduces to
\[
C_{F\text{ideal}}
=
\sqrt{
\frac{2\gamma^2}{\gamma-1}
\left(\frac{2}{\gamma+1}\right)^{\frac{\gamma+1}{\gamma-1}}
\left[
1 - \left(\frac{p_e}{p_c}\right)^{\frac{\gamma-1}{\gamma}}
\right]
} 
\]

From Section~4.3, perfect expansion at sea level for the selected chamber pressure corresponds to:
\[
\frac{p_e}{p_c} \approx 0.0290
\]

Substituting $\gamma = 1.137$ and $p_e/p_c = 0.0290$:
\[
C_{F\text{ideal}}
=
\sqrt{
\frac{2(1.137)^2}{0.137}
\left(\frac{2}{2.137}\right)^{15.60}
\left[1 - (0.0290)^{0.120}\right]
}
\approx 1.52
\]

Thus, for an isentropic, perfectly expanded nozzle with $\varepsilon = 6$ at sea level, the ideal thrust coefficient is:

\[C_{F,\text{ideal}} \approx 1.52\]

Applying the divergence correction factor:

\[C_F = \lambda C_{F,\text{ideal}} = 0.983 \times 1.52 = 1.49\]

This value represents the expected thrust coefficient for an isentropic nozzle with finite divergence operating near sea level, neglecting additional losses such as two phase flow effects and boundary layer growth. The thrust coefficient, characteristic velocity, and specific impulse are related through:

\[I_{sp} g_0 = C_F c^*\]

provided all quantities are defined under consistent assumptions. Using the divergence corrected thrust coefficient $C_F = 1.49$ and the characteristic velocity $c^* = 908~\text{m/s}$:

\[I_{sp,\text{eff}} = \frac{C_F c^*}{g_0}
= \frac{(1.49)(908)}{9.81} \approx 138~\text{s}\]

This effective specific impulse is lower than the ideal literature value $I_{sp} \approx 151.5~\text{s}$, which reflects the inclusion of nozzle divergence losses and the exclusion of additional idealizing assumptions commonly embedded in tabulated $I_{sp}$ values. The value $I_{sp,\text{eff}} \approx 138~\text{s}$ is therefore adopted as the internally consistent performance metric for thrust sizing. As established in previous sections, the thrust produced by the motor is given by:

\[F = C_F p_c A_t\]

From the definition of characteristic velocity: 

\[p_c A_t = \dot m c^*\]

Using the design mass flow rate $\dot m = 0.044~\text{kg/s}$ and
$c^* = 908~\text{m/s}$:

\[p_c A_t = (0.044)(908) = 39.952~\text{N}\]

Substituting into the thrust expression:

\[F = (1.49)(39.952) = 59.5~\text{N}\]

This value is below the minimum thrust requirement of $64~\text{N}$ by approximately $7\%$, indicating that the current mass flow rate and throat sizing are insufficient when evaluated using internally consistent nozzle and combustion performance parameters. The shortfall in predicted thrust does not indicate an inconsistency in the analysis, but rather reflects the transition from idealized impulse based performance estimates to a self consistent nozzle based performance model. The thrust coefficient $C_F = 1.49$ is retained as the expected performance value for the CAT--45 nozzle geometry. Hence, to satisfy the minimum thrust requirement of $F_{\min} = 64~\text{N}$ under the internally consistent nozzle and combustion performance parameters $C_F = 1.49$ and $c^* = 908~\text{m/s}$, the required mass flow rate follows from:

\[F = C_F \dot m c^*\]

Solving for the required mass flow rate:

\[
\dot m_{\text{req}} = \frac{F_{\min}}{C_F c^*}
= \frac{64}{(1.49)(908)}
\]

This value exceeds the previously selected mass flow rate
$\dot m = 0.044~\text{kg/s}$ by approximately $7.5\%$, indicating that the original propellant mass sizing is insufficient when evaluated using the revised performance model.

If the burn duration is maintained at $t_b = 5~\text{s}$, the corresponding required propellant mass is
\[
m_{\text{prop,req}} = \dot m_{\text{req}} t_b
= (0.0473)(5)
= 0.236~\text{kg}
\]

Accordingly, the propellant mass must be increased from the originally selected $220~\text{g}$ to approximately $240~\text{g}$ in order to meet the thrust requirement while preserving the desired burn time and maintaining a sizeable buffer mass of a few grams. This revised propellant mass is adopted for subsequent grain geometry and chamber sizing calculations.

\newpage

\section{Chamber Pressure Selection}

Chamber pressure is a central design parameter in solid rocket motor design  and cannot be selected arbitrarily. It must simultaneously satisfy structural constraints imposed by the motor casing, combustion stability requirements of the propellant, manufacturing limitations, and the fundamental mass flow  balance established in Section 3.2. This section provides the analytical 
justification for the selection of 500 psi as the nominal operating pressure  for the CAT-45 engine.

\subsection{Pressure as a Design Variable}

From Section 3.2, the fundamental constraint linking chamber pressure and  throat area to the required mass flow rate is:

\[
p_c A_t = \dot{m} c^* = 39.952~\text{N}
\]

This relationship reveals that chamber pressure and throat area are not independent, i.e. once one is selected, the other is determined. The selection of chamber pressure therefore establishes the throat diameter through:

\[
A_t = \frac{39.952 \text{ N}}{p_c}
\]

For a given required mass flow rate, \textit{higher} chamber pressure demands \textit{smaller} throat diameter, and vice versa. The optimal chamber pressure  must balance several competing factors:

\begin{enumerate}
\item \textbf{Structural loading:} Higher pressure increases hoop stress in  the motor casing, requiring thicker walls or stronger materials.

\item \textbf{Throat machining precision:} Lower pressure results in larger throat diameter, which is easier to machine accurately on manual equipment.

\item \textbf{Combustion stability:} Too low a pressure may result in unstable or incomplete combustion for the selected propellant.

\item \textbf{Propellant burn rate:} Chamber pressure directly affects propellant burning rate through the pressure-dependent burn rate law, influencing grain geometry and burn time.
\end{enumerate}

The selection process therefore requires simultaneous consideration of structural mechanics, manufacturing capability, and propellant combustion behavior.

\subsection{Structural and Operational Constraints}

The motor casing must safely contain the chamber pressure as a cylindrical pressure vessel. For a thin-walled cylinder, the hoop stress is given by:

\[
\sigma_{\text{hoop}} = \frac{p_c r_i}{t}
\]

where $r_i$ is the internal radius and $t$ is the wall thickness. For 6061-T6 Aluminum, a common material for rocket motor casings:
\begin{itemize}
\item Yield strength: $\sigma_y = 276$ MPa (40030 psi)
\item Ultimate tensile strength: $\sigma_{uts} = 310$ MPa (44962 psi)
\end{itemize}
Al 6061-T6 is selected, both for these properties, and its ease of welding. Steel was also considered for it's strength and ease of welding. However, it was decided against as it releases fragments upon rupture. For this reason most rocketry organizations also ban steel casings on experimental motors. \\

Applying a safety factor of 3 based on yield strength (conservative for 
educational and first-motor applications):
\[
\sigma_{\text{design}} = \frac{\sigma_y}{3} = \frac{276 \text{ MPa}}{3} = 92~\text{MPa}
\]

Standard H-class motors use either 29 mm or 38 mm outer diameter casings. Since 320 Ns is the upper end of H-Class, a 38 mm outer diameter is selected. This sets the outer radius to 19 mm. Rearranging the hoop stress equation to solve for required wall thickness based on a set outer radius:
\[
t = \frac{p_c r_o}{\sigma_{\text{design}}+p_c}
\]

At a chamber pressure of 500 psi (3.447 MPa):
\[
t = \frac{3.447 \text{ MPa} \times 19 \text{mm}}{145 \text{ MPa} + 3.447\text{ MPa}} = 0.69~\text{mm}
\]

This represents the \textit{minimum} theoretical wall thickness. For practical design, accounting for manufacturing tolerances, weld quality variations, potential stress concentrations, and corrosion/erosion during combustion, a wall thickness of
\[
t_{\text{design}} = 2.0~\text{mm}
\]
is specified. This provides an effective safety factor of approximately 2.9 relative to the theoretical minimum, or a safety factor of 8.7 relative to material yield strength. At 2.0 mm wall thickness and 38 mm external diameter, the actual hoop stress  during operation is:
\[
\sigma_{\text{actual}} = \frac{3.447 \text{ MPa} \times (19 \text{ mm} - 2 \text{ mm})}{2\text{ mm}} = 29.3~\text{MPa}
\]

This represents only 10.6\% of the yield strength, providing substantial margin for:
\begin{itemize}
\item Pressure transients during ignition (typically 10-20\% overpressure)
\item Manufacturing defects or weld imperfections
\item Local stress concentrations at nozzle mounting or end closures
\item Thermal effects on material properties during burn
\end{itemize}

For a first motor with manual TIG welding and conservative design practice, chamber pressures are typically limited to 600-800 psi for steel construction. Operating at 500 psi provides comfortable margin below this practical limit while maintaining acceptable structural efficiency. KN/Sorbitol propellant requires a minimum chamber pressure for stable, complete combustion. Below approximately 200-300 psi, several undesirable phenomena may occur:
\begin{itemize}
\item Incomplete combustion of fuel-rich pockets
\item Flame instability and potential extinction
\item Erratic pressure oscillations (chugging)
\item Reduced combustion efficiency and specific impulse
\end{itemize}

Extensive amateur rocketry experience with KN/Sorbitol propellant indicates 
reliable operation above 400 psi. Operating at 500 psi provides a margin of 
25\% above this threshold, ensuring stable combustion throughout the burn. From Section 4.5, the throat diameter at 500 psi is 3.84 mm (0.151 inches). This represents a practical lower limit for precision machining on a manual  lathe:
\begin{itemize}
\item Throat diameter tolerance of $\pm$0.05 mm represents $\pm$1.3\% dimensional control
\item Smaller diameters would require proportionally tighter tolerances
\item Tool deflection and chatter become increasingly problematic at small diameters
\item Measurement and verification of throat geometry becomes more difficult
\end{itemize}

If chamber pressure were reduced to, for example, 300 psi to obtain a larger throat, the resulting diameter would be:

\[
A_t = \frac{39.952}{300 \times 6895} = 1.93 \times 10^{-5}~\text{m}^2
\]
\[
d_t = 2\sqrt{\frac{1.93 \times 10^{-5}}{\pi}} = 4.96~\text{mm}
\]

While this is easier to machine, it approaches the lower end of the acceptable combustion pressure range and reduces structural efficiency (requiring unnecessarily thick walls relative to stress levels). Conversely, if pressure were increased to 700 psi to reduce throat diameter:

\[
A_t = \frac{39.952}{700 \times 6895} = 8.27 \times 10^{-6}~\text{m}^2
\]
\[
d_t = 2\sqrt{\frac{8.27 \times 10^{-6}}{\pi}} = 3.25~\text{mm}
\]

This approaches the practical limit for manual machining precision and increases structural demands on the motor casing.

\subsection{Pressure Margin and Transient Considerations}

During motor ignition, chamber pressure typically exhibits a brief spike (0.1-0.3 seconds) before stabilizing at the steady-state operating value. The magnitude of this transient depends on igniter energy, grain geometry, and propellant characteristics. For conservative design, an ignition overpressure of 20\% is assumed:

\[
p_{\text{max,transient}} = 1.20 \times 500 = 600~\text{psi} = 4.14~\text{MPa}
\]

At this transient pressure, the hoop stress in the 2.0 mm wall becomes:
\[
\sigma_{\text{transient}} = \frac{4.14 \text{ MPa} \times (19\text{ mm} - 2\text{ mm})}{2 \text{mm}} = 35.2~\text{MPa}
\]

This remains well below the design stress of 92 MPa (safety factor of 2.6) and represents only 12.7\% of material yield strength. The motor casing can therefore accommodate expected ignition transients with adequate margin. A common metric in solid rocket motor design is the burst pressure margin, defined as the ratio of predicted burst pressure to maximum expected operating pressure (MEOP). For the CAT-45 motor, \textit{Maximum Expected Operating Pressure:}

\[
\text{MEOP} = 1.20 \times 500 \text{ psi} = 600~\text{psi}
\]

\textit{Predicted Burst Pressure:}
Using ultimate tensile strength and assuming burst occurs at $\sigma = \sigma_{uts}$:
\[
p_{\text{burst}} = \frac{\sigma_{uts} \times t}{r_i} = \frac{310 \text{ MPa} \times 2\text{ mm}}{(19 \text{ mm} - 2\text{ mm})} = 36.5~\text{MPa} = 5290~\text{psi}
\]

\textit{Burst Pressure Margin:}
\[
\text{Margin} = \frac{p_{\text{burst}}}{\text{MEOP}} = \frac{5290 \text{ psi}}{600\text{ psi}} = 8.82
\]

This substantial margin accounts for material variability, weld defects, corrosion, and unforeseen operating conditions. For experimental motors, a burst pressure margin exceeding 4 is considered acceptable; CAT-45 doubles this. During static test operations, chamber pressure must be continuously monitored via a pressure transducer (range: 0-1000 psi, accuracy: $\pm$1\% full scale) mounted in the forward closure. Real-time pressure data serves multiple purposes:
\begin{itemize}
\item Verification of predicted operating pressure
\item Detection of combustion instabilities or anomalies
\item Post-test performance analysis and thrust reconstruction
\item Safety monitoring for abort conditions
\end{itemize}

An automatic pressure-triggered abort system is recommended for static testing, configured to terminate data acquisition and activate fire suppression if chamber pressure exceeds 700 psi (17\% above nominal, 40\% margin below burst).

\newpage

\section{CAT-45 Nozzle Geometry}

Having established the nozzle geometry parameters, expansion ratio, and thrust  coefficient in the preceding sections, the specific dimensional specifications  for the CAT-45 nozzle can now be determined. This section translates the  analytical results into concrete dimensions suitable for fabrication.

\subsection{Throat Area and Diameter Calculation}

From the fundamental constraint established in Section 3.2:

\[p_c A_t = 39.952~\text{N}.\]

Solving for throat area with $p_c = 3.447 \times 10^6$ Pa:

\[A_t = \frac{39.952}{3.447 \times 10^6} = 1.159 \times 10^{-5}~\text{m}^2.\]

The throat diameter is obtained from the circular area relation:

\[
A_t = \pi \left(\frac{d_t}{2}\right)^2
\]

Solving for diameter:
\[
d_t = 2\sqrt{\frac{A_t}{\pi}} = 2\sqrt{\frac{1.159 \times 10^{-5}}{\pi}} = 3.84 \times 10^{-3}~\text{m}
\]

Converting to more convenient units:
\[
d_t = 3.84~\text{mm} = 0.151~\text{inches} = 0.384~\text{cm}
\]

Hence, we can now finalize CAT-45's throat diameter $d_t$ as 3.84  mm or 0.151 inches.

\subsection{Exit Area and Diameter Calculation}

With the expansion ratio $\varepsilon = 6$ established in Section 4.3:
\[
A_e = \varepsilon A_t = 6 \times 1.159 \times 10^{-5} = 6.954 \times 10^{-5}~\text{m}^2
\]

The exit diameter is:
\[
d_e = 2\sqrt{\frac{A_e}{\pi}} = 2\sqrt{\frac{6.954 \times 10^{-5}}{\pi}} = 9.40 \times 10^{-3}~\text{m}
\]

Converting to conventional units:
\[
d_e = 9.40~\text{mm} = 0.370~\text{inches}
\]

Hence, we can now state CAT-45's exit diameter  $d_e$ as 9.40 mm or 0.370 inches.

\subsection{Nozzle Length Calculations}

The axial length of the nozzle consists of two sections: the converging section  upstream of the throat and the diverging section downstream of the throat.\\\

\textbf{Divergent Section Length}: the divergent section extends from the throat to the exit plane at a constant  half-angle $\alpha = 15°$. For a conical divergent section:
\[
L_{\text{div}} = \frac{r_e - r_t}{\tan \alpha}
\]
where $r_e = d_e/2 = 4.70$ mm and $r_t = d_t/2 = 1.92$ mm.

Substituting numerical values:
\[
L_{\text{div}} = \frac{4.70 - 1.92}{\tan(15°)} = \frac{2.78}{0.2679} = 10.4~\text{mm}.
\]\\\

\textbf{Convergent Section Length}: the convergent section connects the combustion chamber to the throat at a  half-angle of $\beta = 45°$, which is typical for solid rocket motor nozzles  and provides smooth flow acceleration with minimal losses. The inlet diameter 
must be selected to match the combustion chamber diameter while ensuring  adequate flow area. For preliminary design, an inlet diameter $d_{\text{inlet}}$ of 25mm is selected. This value will be verified for consistency with the chamber diameter determined from grain geometry in Section 8. The convergent section length is:
\[
L_{\text{conv}} = \frac{r_{\text{inlet}} - r_t}{\tan \beta} = \frac{12.5 - 1.92}{\tan(45°)} = \frac{10.58}{1.0} = 10.6~\text{mm}
\]

The \textbf{total axial length} of the nozzle from inlet plane to exit plane is:

\[
L_{\text{total}} = L_{\text{conv}} + L_{\text{div}} = 10.6 + 10.4 = 21.0~\text{mm}
\]

\textbf{Graphite Stock Requirements}: the nozzle will be machined from a single piece of high-density graphite rod  stock. The required blank dimensions are determined by the maximum nozzle 
diameter and the need for mounting features. The inlet diameter of 25 mm represents the maximum diameter in the nozzle  contour. To allow for external mounting features such as O-ring grooves or  threaded sections, and to provide adequate wall thickness for thermal  protection and structural integrity, a blank diameter of:
\[
d_{\text{blank}} = 32~\text{mm}
\]
is specified. This provides a wall thickness of $(32 - 25)/2 = 3.5$ mm around  the inlet, increasing to $(32 - 9.4)/2 = 11.3$ mm at the exit plane. The required blank length must accommodate the total nozzle length plus  allowance for facing operations, mounting shoulder, and parting-off waste.  A blank length of
\[
L_{\text{blank}} = 40~\text{mm}
\]
is therefore specified. CAT-45's nozzle is designed to be fabricated with high-density, fine-grain graphite (grade ATJ or equivalent) with density  $\rho \approx 1.82$ g/cm$^3$ and compressive strength exceeding 100 MPa. The throat diameter is the most critical dimension in the nozzle and directly  determines mass flow rate and chamber pressure. A tight tolerance is required:

\[
d_t = 3.84 \pm 0.05~\text{mm} \quad (\pm 0.002~\text{inches}).
\]

This tolerance corresponds to approximately $\pm 1.3\%$ on throat diameter,  which translates to $\pm 2.6\%$ on throat area and a similar variation in chamber pressure for a given mass flow rate. Measurement of the as-machined  throat diameter must be performed using precision pin gauges or a coordinate  measuring machine. The exit diameter tolerance is less critical but should be controlled to:
\[
d_e = 9.40 \pm 0.15~\text{mm} \quad (\pm 0.006~\text{inches}).
\]

Divergence and convergence angles should be maintained within $\pm 1°$ of nominal values. Surface finish in the throat region should be better than  3.2 $\mu$m Ra to minimize boundary layer disturbances.

\subsection{Summary of CAT-45 Nozzle Geometry}

The complete dimensional specification for the CAT-45 nozzle is summarized 
in Table 1.

\begin{table}[h]
\centering
\caption{CAT-45 Nozzle Dimensional Specifications}
\begin{tabular}{lcc}
\hline
\textbf{Parameter} & \textbf{Value (mm)} & \textbf{Value (inches)} \\
\hline
Throat diameter & 3.84 $\pm$ 0.05 & 0.151 $\pm$ 0.002 \\
Exit diameter & 9.40 $\pm$ 0.15 & 0.370 $\pm$ 0.006 \\
Inlet diameter & 25.0 & 0.984 \\
Expansion ratio & \multicolumn{2}{c}{6.0} \\
Divergence half-angle & \multicolumn{2}{c}{15°} \\
Convergence half-angle & \multicolumn{2}{c}{45°} \\
Divergent section length & 10.4 & 0.409 \\
Convergent section length & 10.6 & 0.417 \\
Total nozzle length & 21.0 & 0.827 \\
Graphite blank diameter & 32.0 & 1.260 \\
Graphite blank length & 40.0 & 1.575 \\
\hline
\end{tabular}
\end{table}

With the nozzle dimensions now fully specified, the expected thrust can be  verified:
\[
F = C_F p_c A_t = 1.49 \times 3.447 \times 10^6 \times 1.159 \times 10^{-5} = 59.6~\text{N}
\]

This confirms the result obtained in Section 4.4.1, indicating a thrust  approximately 7\% below the target value of 64 N. Close enough. This shortfall can be addressed in subsequent design iterations by adjusting chamber pressure, increasing propellant mass flow rate, or accepting a slightly longer burn 
time to achieve the required total impulse of 320 Ns.

\section{Propellant Description and Characteristics}
\subsection{Potassium Nitrate and Sorbitol Propellant}
\subsection{Combustion Behavior and Thermal Considerations}
\subsection{Suitability for Graphite Nozzle Applications}

\section{Grain Geometry and Burning Surface}
\subsection{Grain Configuration Selection}
\subsection{Burning Area Evolution}
\subsection{Consistency with Chamber Pressure}

\section{Structural Design Considerations}
\subsection{Motor Casing and Pressure Vessel Model}
\subsection{Hoop Stress and Safety Margins}
\subsection{Material Selection Rationale}

\section{Manufacturing Approach}
\subsection{Casing Fabrication Using TIG Welding}
\subsection{Nozzle Machining on Manual Lathe}
\subsection{Dimensional and Alignment Considerations}

\section{References}

\end{document}
